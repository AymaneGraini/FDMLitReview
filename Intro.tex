\documentclass[11pt]{article}
\usepackage{style}
\usepackage{MyNotations}
\begin{document}
\section{Introduction and aim}
\subsection{Introduction}
Modeling plasticity in materials remains today a challenging task due to the interplay of different mechanisms occuring across multiple length scales, which are difficult to integrate into a single computational framework. Plastic deformation in crystalline solids, which manifests macroscopically as permanent deformations, originates from microscopic mechanisms heavily related to dislocations' nucleation, motion, and interactions. Ideally, a perfect crystal plasticty model would encompass relevant mechanisms at each length scale to determine macroscopic mechanical behavior; however, this is computationally unfeasible.\\

Classical plasticity theories in continuum mechanics address material yielding by introducing variables, such as plastic strain and some flow rules, to characterize the onset and evolution of plasticity. While effective for large-scale applications, these models are largely phenomenological; they rely on constitutive laws and parameters that must be empirically tuned or fitted to experimental data. In their heart, crystal plasticty approaches reflect an averaged, macroscopic behavior of dislocation ensembles, inevitably losing subscale information on the underlying microscopic processes. Consequently, classical models often struggle to accurately capture mesoscale effects. In contrast, microscopic theories like \emph{Density Functional Theory} (\textbf{DFT}) \parencite{woodwardPredictionDislocation2008} and \emph{Molecular Dynamics} (\textbf{MD}) \parencite{zhouLargescalemolecular1998} provide detailed descriptions at atomic scales, capturing correct behaviors on short length and time scales. However, these methods are computationally prohibitive at larger time scales and mesoscopic length scales, making them impractical for predicting elasto-plastic responses and microstructural evolution over real-life timescales dimensions.\\
This gap between different models and scales has driven the development of mesoscale theories, aimed at bridging the microscopic and macroscopic scales in crystalline plasticity. Statistical physics introduced various coarse-graining approaches and helped formulating several mesoscale models. Among these, \emph{Discrete Dislocation Dynamics} (\textbf{DDD}), a model that treats dislocations as discrete lines governed by the classical \emph{Peach-Koehler} force, using stress fields derived from continuum mechanics \parencite{devincreThreeDimensionalSimulations1992}. DDD enables the investigation of mechanical properties on mesoscopic time and length scales while explicitly tracking dislocations. However, DDD models still rely on phenomenological prescriptions for dislocation mobility and interaction rules, which can limit their predictive accuracy.\\
To overcome the phenomenological nature of the previous models, \emph{Field Dislocation Mechanics} (\textbf{FDM}) was introduced as a mesoscopic model in which dislocation lines are "smeared out" and represented by a continuous density field \parencite{acharyamodelcrystal2001} . This approach builds on earlier theories of continuously distributed dislocations, and the the dynamical theory of moving dislocations. FDM is a partial differential equation (PDE)-based theory in which atomic vibration kinetic energy is averaged out and exchanged for a dissipative evolution of dislocation density fields, encoded within the mathematical structure of the governing PDEs \parencite{acharyaNewinroads2010}. Furthermore, This PDE framework allows FDM to bypass the need for ad hoc rules for dislocations behavior and interactions. The FDM theory has a robust foundation, as it naturally reproduces several fundamental physical features of plasticity and dislocation mechanics \parencite{zhangsingletheory2015}. However, its primary limitation lies in the absence of crystallographic information within the free energy therein used. Without any crystalline information, the system’s free energy remains convex, meaning that no restoring stresses act on dislocations to keep their cores compact, and no  particular strain direction is energetically favored.\newline

Within the realm of mesoscale and coarse-grained modeling, a widely used technique is the \emph{Phase-Field} (\textbf{PF}) models as offers a robust framework for scale bridging. These models describe distinct phases using continuous order parameters, allowing for an implicit representation of interfaces between them. In particular, the \emph{Phase Field Crystal} (\textbf{PFC}) model provides a mesoscale approach to capture the nonequilibrium dynamics of defected crystalline materials while accouting for elasticity \parencite{elderModelingElasticity2002}, \parencite{elderModelingelastic2004}. It introduces a scalar field $\psi$, representing atomic density, alongside a free energy functional tailored to be minimized with the desired lattice symmetry. The evolution of this field is governed by dissipative conserved dynamics. In PFC, atomic vibrations where averaged out making the model valid only at diffusive timescales, similarily to FDM. However, PFC has the advantage of naturally incorporating topological defects and their evolution, the later being a common occurence in symmetry breaking phase transitions \parencite{anghelutaAnisotropicvelocity2012}, and can be easily identfied within the amplitude expansion limit. As such, it has been widely used to study dislocations in crystals \parencite{skogvollphasefield2022}. Importantly, and due to its diffusive timescale, the PFC model struggles to capture elasticity and other wave-like behaviors accurately. The main issue being that the model rlies on a single scalar field to describe both atomic density and lattice distortion and it links the evolution of these variables to the slower, diffusive dynamics of the scalar field, which results in an unphyiscal behavior \parencite{acharyaElasticityphase2022}. To address these limitations, various enhancements have been proposed in the literature. These include adding a second time derivative to the evolution equation \parencite{stefanovicPhaseFieldCrystals2006}, introducing a dissipative current, displacing the scalar field with a compatible distortion to enforce equilibrium \parencite{skaugenSeparationElastic2018}, and even developing a full PFC-hydrodynamics theory \parencite{skogvollHydrodynamicphase2022}. Nonetheless, these attempts often incorporate elasticity as a second step, or even like an afterthought.\\

In this work, we explore a novel approach to address the limitations of both PFC and FDM. We mentioned that, the PFC model accurately captures topological defects and their motion but lacks the ability to model elasticity, while FDM successfully models elasticity and dislocation mechanics but lacks crystallographic information. Thus, it seemed only natural to attempt a synthesis of the two theories to develop an ad hoc, physically accurate mesoscale model for dislocations. This approach, proposed in \parencite{acharyaFielddislocation2020}, is inspired by the \emph{Peierls-Nabarro} model: we add a non-convex term in the free energy of the system, using the \emph{Swift-Hohenberg} functional used in PFC modeling. A coupled FDM-PFC theory is then formulated and in which the scalar field $\psi$ does not contribute to the elastic energy and is not interpreted as a mass density. Instead, the phase field serves solely as an indicator of defects, with its topological properties coupled to the elasticity of medium. By explicitly separating mass' motion and defect's motion, we investigate wether the new model will yield a correct framework to study dislocations' motion and plastity in crystalline material.\\

\subsection{Motivation and aim}
Update from presentation
\end{document}